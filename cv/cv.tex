% LaTeX Curriculum Vitae Template
%
% Copyright (C) 2004-2009 Jason Blevins <jrblevin@sdf.lonestar.org>
% http://jblevins.org/projects/cv-template/
%
% You may use use this document as a template to create your own CV
% and you may redistribute the source code freely. No attribution is
% required in any resulting documents. I do ask that you please leave
% this notice and the above URL in the source code if you choose to
% redistribute this file.

\documentclass[A4]{article}

\usepackage{hyperref}
\usepackage[left=2cm, right=2cm, top=2.5cm, bottom=2.5cm]{geometry}

% Comment the following lines to use the default Computer Modern font
% instead of the Palatino font provided by the mathpazo package.
% Remove the 'osf' bit if you don't like the old style figures.
\usepackage[T1]{fontenc}
\usepackage[sc,osf]{mathpazo}
\usepackage{fontspec}		%XeLaTeX字型的package
\newfontfamily{\Kai}[Mapping=tex-text]{AR PL UKai TW MBE} % \Kai切換字型為標楷體
\newfontfamily{\Hei}[Mapping=tex-text]{Noto Sans CJK TC} % \Hei切換字型為黑體 
\newfontfamily{\Sung}[Mapping=tex-text]{Noto Serif CJK TC} % \Song切換字型為宋體
\setmonofont[Mapping=tex-text]{Source Code Pro} % 等寬字型
\setromanfont[Mapping=tex-text]{Source Serif Pro} % Roman字型
%\newfontfamily{\H}[Mapping=tex-text]{HeiTi} % \K切換字型為標楷體

% Set your name here
\def\name{Min-Zheng Shieh {\Kai 謝旻錚}}

% Replace this with a link to your CV if you like, or set it empty
% (as in \def\footerlink{}) to remove the link in the footer:
\def\footerlink{}

% The following metadata will show up in the PDF properties
\hypersetup{
  colorlinks = true,
  urlcolor = black,
  pdfauthor = {\name},
  pdfkeywords = {computer science, information science, mathematics},
  pdftitle = {\name: Curriculum Vitae},
  pdfsubject = {Curriculum Vitae},
  pdfpagemode = UseNone
}

%\geometry{
%  body={6.5in, 8.5in},
%  left=1.0in,
%  top=1.25in
%}

% Customize page headers
\pagestyle{myheadings}
\markright{\name}
\thispagestyle{empty}

% Custom section fonts
\usepackage{sectsty}
\sectionfont{\rmfamily\mdseries\Large}
\subsectionfont{\rmfamily\mdseries\itshape\large}

% Other possible font commands include:
% \ttfamily for teletype,
% \sffamily for sans serif,
% \bfseries for bold,
% \scshape for small caps,
% \normalsize, \large, \Large, \LARGE sizes.

% Don't indent paragraphs.
\setlength\parindent{0em}

% Make lists without bullets
\renewenvironment{itemize}{
  \begin{list}{}{
    \setlength{\leftmargin}{1.5em}
  }
}{
  \end{list}
}

\begin{document}

% Place name at left
{\huge \name}

% Alternatively, print name centered and bold:
%\centerline{\huge \bf \name}

\vspace{0.25in}

\begin{minipage}{0.45\linewidth}
  CS 337\\
	\href{http://www.it.nctu.edu.tw/}{Information Technology Service Center}\\
  \href{http://www.nctu.edu.tw/}{National Chiao Tung University} \\
  1001 University Road, Hsinchu City, Taiwan 300
\end{minipage}
\begin{minipage}{0.45\linewidth}
  \begin{tabular}{ll}
    Phone: & +886-3-5712121 EXT 31265 \\
    Fax: & +886-3-5714031 \\
    Email: & \href{mailto:mzshieh@nctu.edu.tw}{\tt mzshieh@nctu.edu.tw} \\
    Homepage: & \href{http://sites.google.com/site/mzshieh}{\tt https://sites.google.com/site/mzshieh} \\
  \end{tabular}
\end{minipage}

\section*{Research Interests}
\begin{itemize}
\item Computational complexity
\item Approximation
\item Algorithms
\item Combinatorics
\item Coding theory
%\item Game theory
\end{itemize}

%\section*{Personal}

%\begin{itemize}
%\item Born on December 1, 1980.
%\item Taiwan Citizen.
%\end{itemize}


\section*{Education}

\begin{itemize}

  \item Ph.D. in Computer Science and Engineering of National Chiao Tung University (2011)
  \item M.S. in Computer Science and Information Engineering of National Chiao Tung University (2004)
  \item B.S. in Computer Science and Information Engineering of National Chiao Tung University (2003)
 
\end{itemize}

\section*{Work Experience}

\subsection*{Academic Positions}
\begin{itemize}
\item 2016-present: Assistant professor of the Information Technology Service Center, NCTU
\item 2012-2016: Assistant research fellow of the Information and Communication Technology Laboratories, NCTU
\item 2011-2012: Postdoctoral researcher of the Information and Communication Technology Laboratories, NCTU
\end{itemize}
\subsection*{Course Taught}
\begin{itemize}
\item 2018 Spring: Competitive Programming (I), Department of Computer Science, NCTU
\item 2018 Spring: Scratch and Python, Information Technology Service Center, NCTU
\item 2017 Fall: Competitive Programming (II), Department of Computer Science, NCTU
\item 2017 Fall: Scratch and Python, Information Technology Service Center, NCTU
\item 2017 Summer: Scratch and Python, Information Technology Service Center, NCTU
\item 2017 Spring: Competitive Programming (I), Department of Computer Science, NCTU
\item 2017 Spring: Scratch and Python, Information Technology Service Center, NCTU
\item 2016 Fall: Competitive Programming (III), Department of Computer Science, NCTU
\item 2016 Fall: Scratch and Python, Information Technology Service Center, NCTU
\item 2016 Summer: Scratch and Python, Information Technology Service Center, NCTU
\item 2016 Spring: Competitive Programming (II), Department of Computer Science, NCTU
\item 2015 Fall: Competitive Programming (III), Department of Computer Science, NCTU
\item 2015 Fall: Competitive Programming (I), Department of Computer Science, NCTU
\item 2015 Spring: Competitive Programming (II), Department of Computer Science, NCTU
\item 2014 Fall: Competitive Programming (I), Department of Computer Science, NCTU
\item 2014 Spring: Introduction to Algorithms, Department of Computer Science, NCTU
\item 2014 Spring: Problem Solving and Programming Techniques, Department of Computer Science, NCTU
\item 2013 Summer: Data Structures (Honors), Department of Computer Science, NCTU
\item 2013 Spring: Advanced Computer Programming, Institute of Computer Science and Engineering, NCTU
\item 2012 Summer: Advanced Competitive Programming (III), Institute of Computer Science and Engineering, NCTU
\item 2012 Spring: Advanced Computer Programming, Institute of Computer Science and Engineering, NCTU
\item 2010 fall: Lectures in Computer Science (III), Department of Computer Science, NCTU
\item 2010 fall: Lectures in Computer Science (I), Department of Computer Science, NCTU
\end{itemize}
\subsection*{Research Assistant}
\begin{itemize}
\item 2006-2009: Mitac's project on ``Route Planning for Navigation System,'' led by Prof. Shi-Chun Tsai
\item 2006-2009: National Science Council's project on ``Probabilistic Proof Systems, Inapproximability, and Expander Graphs", led by Prof. Shi-Chun Tsai
\end{itemize}
\subsection*{Websites}
\begin{itemize}
\item 2007-2008: Technology consultant of UbicTek ({\Sung 長春吉科技})
\item 2001-2003: Administrator of~~{\tt bbs.wretch.cc} {\Sung 無名小站} (acquired by Yahoo, 2007)
\end{itemize}
\subsection*{Teaching Assistant}
\begin{itemize}
\item 2011 spring: Introduction to Algorithms
\item 2010 spring: Introduction to Algorithms
\item 2009 fall: Formal Languages and Theory of Computation
\item 2009 summer: Competitive Programming(I)
\item 2009 spring: Introduction to Algorithms
\item 2007 fall: Advanced Programming and Algorithms
\item 2007 spring: Theory of Computation
\item 2006 fall: Introduction to algorithms
\item 2006 spring: Combinatorics
\item 2005 fall: Introduction to algorithms
\item 2005 spring: Formal language
\item 2004 fall: Algorithms
\item 2004 spring: Problem Solving Technics
\item 2003 fall: Introduction to Computer Science
\end{itemize}

\subsection*{Programming Contest}
\begin{itemize}
\item 2014: Lecturer, Taiwanese Olympiads of Informatics
\item 2012-present: Coach of National Chiao Tung University ACM ICPC Teams
\item 2006-2011: Assistant coach of National Chiao Tung University ACM ICPC Teams
\end{itemize}

\section*{Honors}
\subsection*{Memberships and Scholarships}
\begin{itemize}
\item Honorable Member of The Phi Tau Phi Scholastic Honor Society of The Republic of China, 2012
(Top $10\%$ of the doctoral graduates)
\item Scholarship of Elite Teaching Assistant, two times, 2007 and 2009
\item Scholarship of College of Electric Engineering and Computer Science, 2003
\item Honorable Member of The Phi Tau Phi Scholastic Honor Society of The Republic of China, 2003
(Top $1\%$ of the graduates)
\end{itemize}

\subsection*{Awards}
\subsubsection*{Contestant}
\begin{itemize}
\item Academic Achievement Awards in NCTU CSIE, 5 times during 1999-2003
\item 4th place in ACM Intercollegiate Programming Contest Kaohsiung Site, 2002
\item 6th place in ACM Intercollegiate Programming Contest Taipei Site, 2001
\item 2nd place in National Collegiate Programming Contest, 2001
\item 5th place in Trend Micro Million Programming Contest, 2001
\item 3rd place in Internet Problem Solving Contest, 2001
\item 5th place in ACM Intercollegiate Programming Contest Taipei Site, 2000
\item Silver medalist in International Olympiads in Informatics, 1999
\end{itemize}
\subsubsection*{Coach}
\begin{itemize}
\item 2nd place in National Collegiate Programming Contest, 2017
\item 2nd place in National Collegiate Programming Contest, 2016
\item 44th place in ACM International Collegiate Programming Contest World Finals, 2016
\item 2nd place in ACM International Collegiate Programming Contest Jakarta Regional, 2015
\item 5th place in ACM International Collegiate Programming Contest Taipei Regional, 2015
\item 7th place in ACM International Collegiate Programming Contest Phuket Regional, 2015
\item 9th place in ACM International Collegiate Programming Contest Daejeon Regional, 2015
\item 2nd place in National Collegiate Programming Contest, 2015
\item 6th place in ACM International Collegiate Programming Contest Taichung Regional, 2014
\item 9th place in ACM International Collegiate Programming Contest Daejeon Regional, 2014
\item 3rd place in National Collegiate Programming Contest, 2014
\item 12th place in ACM International Collegiate Programming Contest Daejeon Regional, 2014
\item 12th place in ACM International Collegiate Programming Contest Chiayi Regional, 2013
\item 3rd place in National Collegiate Programming Contest, 2013
\item 9th place in ACM International Collegiate Programming Contest Kaohsiung Regional, 2012
\item 3rd place in National Collegiate Programming Contest, 2012
\end{itemize}

\section*{Publications}

\subsection*{Journal papers}

\begin{enumerate}
\item Min-Zheng Shieh, Shi-Chun Tsai, ``Inapproximability results for the weight problems of subgroup permutation codes,'' IEEE Transactions on Information Theory (SCI), Vol. 58(11), pp. 6907--6915, 2012. (JCR impact factor 2011: 3.009)
\item Min-Zheng Shieh, Shi-Chun Tsai, Ming-Chuan Yang, ``On the Inapproximability of Maximum Intersection Problems,'' Information Processing Letters (SCI), Vol. 112(19), pp. 723--727, 2012. (JCR impact factor 2011: 0.455)
\item Min-Zheng Shieh, Shi-Chun Tsai, ``Computing the Ball Size of Frequency Permutations under Chebyshev Distance,'' Linear Algebra and its Applications (SCI), Vol. 437(1), pp. 324--332, 2012. (JCR impact factor 2011: 0.974)
\item Chia-Jung Lee, Te-Tsung Lin, \emph{Min-Zheng Shieh}, Shi-Chun Tsai, Hsin-Lung Wu, ``Decoding Permutation Arrays with Ternary Vectors,'' Designs, Codes and Cryptography (SCI), Vol. 61(1), pp. 1--9, 2011. (JCR impact factor 2011: 0.875)
\item Li-Jui Chen, Jinn-Jy Lin, \emph{Min-Zheng Shieh}, Shi-Chun Tsai,
``More on the Derek-Magnus Game,''  Theoretical Computer Science (SCI), Vol. 412, pp. 339--344, 2011. 
(JCR impact factor 2011: 0.665)
\item \emph{Min-Zheng Shieh} and Shi-Chun Tsai, ``Improved Bound on Approximating Jug Measuring Problem,'' Journal of Information Science and Engineering (SCI), Vol. 27, No. 3, pp. 1159--1163, 2011. 
(JCR impact factor 2011: 0.175)
\item \emph{Min-Zheng Shieh} and Shi-Chun Tsai, ``Decoding Frequency Permutation Arrays under Chebyshev distance,'' IEEE Transactions on Information Theory (SCI), Vol 56(11), pp. 5730--5737, 2010.
(JCR impact factor 2010: 2.725)
\item \emph{Min-Zheng Shieh} and Shi-Chun Tsai, ``Jug Measuring: Algorithms and Complexity,'' Theoretical Computer Science (SCI), Vol 396, pp. 50--62, 2008.
(JCR impact factor 2010: 0.838)
\item Ying-Jie Liao, \emph{Min-Zheng Shieh} and Shi-Chun Tsai, ``Arranging Numbers on Circles to Reach Maximum Total Variations,'' the Electronic Journal of Combinatorics (SCI), R47: Volume 14(1), 2007.
(JCR impact factor 2010: 0.626)
\end{enumerate}

\subsection*{Conference papers}

\begin{enumerate}
\item \emph{Min-Zheng Shieh} and Shi-Chun Tsai, ``Computing the Ball Size of Frequency Permutations under Chebyshev Distance,'' 2011 IEEE International Symposium on Information Theory (ISIT 2011), July 31- Augest 5, 2011, Saint Petersburg, Russia.
\item \emph{Min-Zheng Shieh} and Shi-Chun Tsai, ``On the Minimum Weight Problem of Permutation Codes under Chebyshev Distance,'' 2010 IEEE International Symposium on Information Theory (ISIT 2010), June 13- June 18, 2010, Austin, Texas, USA.
\item \emph{Min-Zheng Shieh} and Shi-Chun Tsai, ``Decoding Frequency Permutation Arrays under Infinite Norm,'' 2009 IEEE International Symposium on Information Theory (ISIT 2009), June 28- July 3, 2009, Seoul, Korea.
\item Ming Yu-Hsieh, \emph{Min-Zheng Shieh}, Shi-Chun Tsai, ``A Greedy Approach to the Coin Exchange Problem,'' International Computer Symposium 2004, Tainan, Taiwan.
\end{enumerate}

\bigskip

% Footer
\begin{center}
  \begin{footnotesize}
    Last updated: \today \\
    \href{\footerlink}{\texttt{\footerlink}}
  \end{footnotesize}
\end{center}

\end{document}
